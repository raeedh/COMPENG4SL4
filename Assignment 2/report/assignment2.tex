\documentclass[12pt]{article}
\usepackage[letterpaper, margin=1in]{geometry}
\usepackage{graphicx}
\usepackage{subcaption}
\graphicspath{{./../figures/}}
\usepackage{hyperref}
\hypersetup{colorlinks=true}
\usepackage{parskip}
\usepackage{amsmath}
\usepackage{siunitx}
\usepackage{pgfplots}
\usepackage{pgfplotstable}
\pgfplotsset{compat = newest}
\usepackage{titlesec}
\usepackage{listings}
\usepackage{lstautogobble}
%\usepackage[framed, numbered]{matlab-prettifier}
\lstset{inputpath=../../code,frame=single,breaklines=true,numbers=left}

\titleformat*{\section}{\large\bfseries}
%\allowdisplaybreaks

% remove vertical spacing above top figure
\makeatletter
\setlength{\@fptop}{0pt}
\makeatother
%

\title{COMPENG 4SL4 Assignment 2 Report}
\author{
    Raeed Hassan \\
    hassam41 \\
    400188200
    %L02
}

\begin{document}

\maketitle
\clearpage

\section*{Data Set Loading and Splitting}
The data set was loaded using \texttt{sklearn.datasets.load\_boston} and split into a training set (containing 80\% of the data sample points), and a test set (containing the other 20\% of the data sample points). The \texttt{sklearn.model\_selection.train\_test\_split} function was used to perform the split, using the last 4 numbers of my student ID (8200) as the random state for the function.

\section*{K-Fold Cross-Validation Setup}
For K-fold cross-validation in this lab, the value of $K$ was set to 5. Arrays containing the training and test indexes for each fold were created using \texttt{sklearn.model\_selection.KFold}. These indices are used for all K-Fold cross-validation done throughout the lab. These indices are used for row indexing to generate the feature and target matrices before doing regression.

\section*{Feature Selection with no Basis Expansion}
K-Fold cross-validation is performed for remaining feature in a loop until all features have exhausted. The error for each fold is calculated, then the errors are averaged to calculate the cross-validation error for each feature tested. The feature will the lowest cross-validation error is selected as the best feature. The model is then retrained with the best feature using the entire training set and the test error is calculated using the test set. 

Table~\ref{tab:no_basis} shows the best feature for $k$ = 1--13, and the cross-validation and test errors when adding this feature to $S$. For every $k$, when using the best feature the cross-validation error was smaller than the test error. This relation also largely scaled with the cross-validation error, with a larger cross-validation error generally corresponding to a larger test error. The lowest cross-validation error was recorded for $k = 11$, which had the third lowest test error. The largest cross-validation error at $k = 1$ also had the largest test error. The cross-validation and test errors are plotted on Figure~\ref{fig:error}.

\section*{Feature Selection with Basis Expansion}
When the best feature for any $k$ was chosen, K-fold cross-validation basis expansion was performed against but with basis expansion performed on the feature. The two basis functions chosen were selected to compare two functions that would generally scale down the values of all the features. The two basis functions selected were $\phi_k(x) = \sqrt{x_j}$ and $\phi_k(x) = \ln(x_j + 1)$. The second basis function had modified to include the addition as many of the values in feature $f_{(4)}$ contain zeros, and this addition barely affected the cross-validation error when tested. The cross-validation errors using the two basis functions were compared, and the function that produced the lower cross-validation error was selected for every $k$, and the model is then retrained with the best feature and basis expansion using the entire training set and the test error is calculated using the test set.

Table~\ref{tab:basis} shows the best feature for $k$ = 1--13 with basis expansion, and the cross-validation errors with both basis functions, and test error with the best basis function. For all values of $k$, the basis function, $b_2$, which is $\phi_k(x) = \ln(x_j + 1)$ outperformed $b_1$ with a smaller cross-validation error. Similar to the case with no basis expansion, when the model is retrained with basis expansion using the training set and the test error is calculated, the cross-validation error is smaller than the test error for every $k$. This relation also largely scaled with the cross-validation error, with a larger cross-validation error generally corresponding to a larger test error. The lowest cross-validation error was recorded for $k = 13$, which had the lowest test error. The largest cross-validation error at $k = 1$ also had the largest test error. The cross-validation and test errors are plotted on Figure~\ref{fig:error}.

\begin{table}[htp]
\centering
\caption{Errors for Best Feature for $k$ = 1--13}\label{tab:no_basis}
\begin{tabular}{|l|l|l|l|}
	\hline
	$k$	& Best Feature	& Cross-Validation Error & Test Error         \\ \hline\hline
	1  	& $f_{(13)}$	& 37.52981349037695      & 44.01642965365188  \\ \hline
	2  	& $f_{(6)}$		& 30.232181061635934     & 34.84114009877059  \\ \hline
	3  	& $f_{(11)}$	& 26.93968194188119      & 32.20558361910458  \\ \hline
	4  	& $f_{(8)}$		& 26.177187358856212     & 30.645130159002235 \\ \hline
	5  	& $f_{(5)}$		& 24.912879835977943     & 28.286173753054985 \\ \hline
	6  	& $f_{(12)}$	& 24.30481665830434      & 28.388570169930453 \\ \hline
	7  	& $f_{(2)}$		& 24.045578821916497     & 27.829372049292544 \\ \hline
	8  	& $f_{(1)}$		& 23.93238097885264      & 27.62940695273575  \\ \hline
	9  	& $f_{(9)}$		& 23.553279587919693     & 27.145745012675526 \\ \hline
	10 	& $f_{(10)}$	& 22.87754245924457      & 26.15100337729019  \\ \hline
	11 	& $f_{(3)}$		& 22.936063395709063     & 26.10522648390962  \\ \hline
	12 	& $f_{(7)}$		& 23.0123559430166       & 26.203337950256564 \\ \hline
	13 	& $f_{(4)}$		& 23.17317153083585      & 25.081388777344092 \\ \hline
\end{tabular}
\end{table}

\begin{table}[htp]
\centering
\caption{Errors for Best Feature for $k$ = 1--13 with Basis Expansion}\label{tab:basis}
\resizebox{\textwidth}{!}{
\begin{tabular}{|l|l|l|l|l|}
	\hline
	k	& Cross-Validation Error for $b_1$ & Cross-Validation Error for $b_2$ & Best Basis Function & Test Error \\ \hline\hline
	1	& 31.804778028535747 & 28.78911747132245  & $b_2$ & 30.26744238122664  \\ \hline
	2	& 27.82424066223519  & 26.18837863475094  & $b_2$ & 28.382217037889895 \\ \hline
	3	& 25.084940264392877 & 23.804041380557386 & $b_2$ & 26.5603133022723   \\ \hline
	4	& 23.511570217312173 & 22.09032521586629  & $b_2$ & 23.83276391329117  \\ \hline
	5	& 22.227661019873914 & 20.738081968392216 & $b_2$ & 21.77507921513151  \\ \hline
	6	& 21.619810645158452 & 20.130119743750072 & $b_2$ & 21.813762402597206 \\ \hline
	7	& 21.468492327469754 & 20.088725383807375 & $b_2$ & 21.900783544904453 \\ \hline
	8	& 21.551075676436437 & 20.192849446212882 & $b_2$ & 21.897733955789434 \\ \hline
	9	& 20.86242873106243  & 19.297245284435142 & $b_2$ & 22.02694846079194  \\ \hline
	10	& 20.10985039649996  & 18.68610785258567  & $b_2$ & 21.07789265155987  \\ \hline
	11	& 20.141963029677022 & 18.653648609589492 & $b_2$ & 20.91134050704141  \\ \hline
	12	& 20.096759819961267 & 18.54072504399405  & $b_2$ & 21.115487227577024 \\ \hline
	13	& 20.16513360676659  & 18.531483796143725 & $b_2$ & 20.077605226890636 \\ \hline
\end{tabular}
}
\end{table}

\clearpage
\section*{Cross-Validation and Test Error Plot}
Figure~\ref{fig:error} shows the cross-validation and test errors before and after basis expansion, with the errors for no basis expansion marked with circles and errors with basis expansion marked with crosses. We can see that performing basis expansion decreases both the cross-validation and test errors for the model. We can also see that the cross-validation errors are smaller than the test errors in both cases, as previously discussed.

\begin{figure}[htp]
\centering
\begin{tikzpicture}
	\begin{axis}[
		title = {Cross-Validation and Test Errors vs. $k$},
		width = \textwidth,
		%height = 150,
		xmin = 0.5, xmax = 13.5,
		ymin = 0, ymax = 50,
		ylabel = Error, xlabel = $k$,
		xticklabel style={
		/pgf/number format/fixed,
		/pgf/number format/precision=3
		},
		scaled x ticks=false,
		legend pos = north east,
		% some code for adding points
		%nodes near coords={%
		%\footnotesize
		%$(\pgfmathprintnumber
		%{\pgfkeysvalueof{/data point/x}},
		%\pgfmathprintnumber
		%{\pgfkeysvalueof{/data point/y}})$%
		%},
	]
	\addplot [only marks, mark=o, color=blue] table [y=no_basis_cross, x=$k$] {./data/error.dat};
	\label{plot:no_basis_cross}
	\addlegendentry{Cross-Validation Error with no Basis Expansion}

	\addplot [only marks, mark=o, color=red] table [y=no_basis_test, x=$k$] {./data/error.dat};
	\label{plot:no_basis_test}
	\addlegendentry{Test Error with no Basis Expansion}

	\addplot [only marks, mark=x, color=green] table [y=basis_cross, x=$k$] {./data/error.dat};
	\label{plot:basis_cross}
	\addlegendentry{Cross-Validation Error with Basis Expansion}

	\addplot [only marks, mark=x, color=black] table [y=basis_test, x=$k$] {./data/error.dat};
	\label{plot:basis_test}
	\addlegendentry{Test Error with Basis Expansion}

	\end{axis}
\end{tikzpicture}
\caption{Cross-Validation and Test Errors vs. $k$}
\label{fig:error}
\end{figure}

\end{document}